
\documentclass[10pt,a5paper]{article}

% Encoding, fonts, language
\usepackage[T2A]{fontenc}
\usepackage[utf8]{inputenc}
\usepackage[english, russian]{babel}
\usepackage{fontspec}
\setmainfont{Times New Roman}

% Geometry and spacing
\usepackage[left=20mm,right=20mm,top=20mm,bottom=20mm]{geometry}
\usepackage{setspace}
\onehalfspacing
\usepackage{microtype}
\usepackage{titlesec}
\titleformat{\section}[block]{\centering\Large\bfseries}{}{0em}{}
\titleformat{\subsection}[block]{\large\bfseries}{}{0em}{}
\titleformat{\subsubsection}[block]{\large\bfseries}{}{0em}{}

% Math & symbols
\usepackage{amsmath,amssymb,amsfonts}

% Graphics
\usepackage{graphicx}
\graphicspath{{media/}}
\usepackage{float}
\usepackage{caption}
\captionsetup[figure]{font=small}
\usepackage{subcaption}

% Tables
\usepackage{array,booktabs,longtable,tabularx}

% References and links
\usepackage[hidelinks]{hyperref}


% Number figures/tables per section
\numberwithin{figure}{section}
\numberwithin{table}{section}

% Paragraph formatting
\setlength{\parindent}{0.75cm}
\setlength{\parskip}{0pt}

% Улучшение выравнивания для русского языка
\XeTeXlinebreaklocale "ru"
\XeTeXlinebreakskip = 0pt plus 1pt

\begin{document}

\begin{center}
\large
Соловьев П.А., Бахмуров А.Г.

{\bfseries\Large Исследование и разработка алгоритма выравнивания изолинии электрокардиограммы}
\end{center}

% ---------------- Main text ----------------
\section{Введение}
Электрокардиография (ЭКГ) является одним из ключевых методов диагностики сердечно-сосудистых заболеваний, позволяя регистрировать электрическую активность сердца и выявлять патологии его работы. Сердечно-сосудистые заболевания остаются ведущей причиной смертности во всем мире: согласно данным Всемирной организации здравоохранения, они ежегодно уносят жизни более 17 миллионов человек, что составляет около 31\% всех случаев смерти~\cite{worldhealth}. Своевременная и точная диагностика с использованием ЭКГ играет критическую роль в снижении этих показателей, предоставляя врачам информацию о ритме, проводимости и состоянии сердечной мышцы.

Развитие технологий в области электрокардиографии существенно расширило возможности диагностики и мониторинга сердечно-сосудистых заболеваний. Современные ЭКГ-устройства эволюционировали от стационарных аппаратов до портативных и носимых гаджетов, таких как смарт-часы и патч-мониторы для ЭКГ, которые обеспечивают непрерывный мониторинг сердечной активности в реальном времени. Значительный вклад в эту область вносит Интернет вещей (IoT), позволяющий интегрировать ЭКГ-устройства в единую экосистему для сбора, передачи и анализа данных. IoT-системы обеспечивают удаленный доступ к ЭКГ-сигналам, их автоматическую обработку с использованием алгоритмов искусственного интеллекта и передачу результатов врачам через облачные платформы. Это не только повышает доступность диагностики, особенно в удаленных регионах, но и способствует раннему выявлению патологий, минимизируя риски осложнений.

Однако качество сигналов ЭКГ часто снижается из-за помех, таких как дрейф изолинии \textendash\ низкочастотное смещение базового уровня, вызванное движением пациента, дыханием или нарушением контакта электродов. Это искажение затрудняет анализ ЭКГ, особенно критически важного ST-сегмента, используемого для диагностики ишемии и инфаркта миокарда.

Таким образом, разработка эффективных алгоритмов коррекции изолинии остается актуальной задачей. Устранение дрейфа позволяет: улучшить визуализацию сигнала для врача; избежать маскировки или ложного обнаружения патологий; обеспечить корректную работу автоматических алгоритмов анализа; повысить точность диагностики в мобильных ЭКГ-системах.

Данная статья посвящена исследованию и разработке алгоритма выравнивания изолинии электрокардиограммы, который позволит минимизировать дрейф изолинии.

Эта работа имеет следующую стуктуру. В первом разделе выделены критерии сравнения методов удаления дрейфа изолинии и проведен обзор основных существующих методов. Во втором разделе более подробно описан выбранный алгоритм. В третьем разделе приведен план экспериментального исследования и его результаты.

\section{1. Обзор адаптивных алгоритмов удаления дрейфа изолинии}
В данном разделе проводится обзор наиболее популярных адаптивных методов выравнивания изолинии сигнала ЭКГ. Они сравниваются по выделенным далее критериям и на основе сравнения выбирается метод для дальнейшей реализации. Цель обзора \textendash\ выявить его преимущества и недостатки, а также область применимости.

\subsection{1.1 Критерии сравнения}
\begin{itemize}
  \item \textbf{Улучшение SNR (дБ)} \textendash\ показывает, насколько фильтр увеличивает отношение сигнал-шум (Signal-to-Noise Ratio, SNR). Чем выше значение, тем лучше фильтр подавляет шум, сохраняя полезный сигнал.
  \item \textbf{Искажение сигнала (RMSE, \%)} \textendash\ отражает уровень изменений исходного сигнала после фильтрации.
  \item \textbf{Итерации до сходимости} \textendash\ количество итераций (обновлений весов фильтра), необходимых для достижения стабилизации ошибки фильтрации при постоянных характеристиках шума.
  \item \textbf{Сложность на отсчёт} \textendash\ вычислительная сложность алгоритма на каждом шаге.
  \item \textbf{Наличие открытой программной реализации} \textendash\ упрощает разработку и тестирование алгоритма.
\end{itemize}

\subsection{1.2 Обзор методов}
На основе анализа литературы было выделено семь методов, наиболее распространённых в задаче удаления дрейфа изолинии сигнала ЭКГ.

\subsubsection{1.2.1 LMS (Least Mean Squares)}
Метод наименьших средних квадратов~\cite{pandey2010} является эффективным инструментом для адаптивной фильтрации сигналов, включая задачу удаления дрейфа изолинии. Он часто применяется как часть более сложных методов, включающих предварительную оценку частоты дрейфа изолинии.

\textbf{Выход фильтра}
\begin{equation*}
  \hat{y}[n] = \sum_{i=0}^{M-1} w_i(n)\, x[n-i].
\end{equation*}
\noindent
Где \(\hat{y}[n]\) \textendash\ предсказанный (оценённый) выход, \(x[n]\) \textendash\ входной сигнал,
\(w_i(n)\) \textendash\ весовые коэффициенты адаптивного фильтра на отсчёте n, \(M\) \textendash\ длина окна (порядок фильтра).

\medskip
\textbf{Ошибка фильтрации}
\begin{equation*}
  e[n] = d[n] - \hat{y}[n],
\end{equation*}
\noindent
где \(d[n]\) \textendash\ желаемый (эталонный) сигнал.

\medskip
\textbf{Правило обновления весов}
\begin{equation*}
  w_i\ (n+1)=w_i\ (n)+2\mu\cdot e(n)\cdot x(n-i)
\end{equation*}
\noindent
где \(\mu > 0\) \textendash\ шаг обучения (скорость адаптации).


Алгоритм LMS работает в реальном времени, постепенно подстраивая коэффициенты фильтра для уменьшения среднеквадратичной ошибки. Преимущество LMS в его простоте и низкой вычислительной стоимости, что делает его подходящим для обработки сигналов в реальном времени. Основной недостаток \textendash\ чувствительность к шагу обучения $\mu$: при слишком большом значении алгоритм может стать нестабильным, при слишком малом \textendash\ будет сходиться медленно. Также он требует эталонного сигнала, что обычно недоступно в реальных задачах.

\subsubsection{1.2.2 NLMS (Normalized Least Mean Squares)}
Этот метод~\cite{rehman2012} является модификацией LMS. Выход фильтра и ошибка фильтрации совпадают с LMS алгоритмом, но отличие заключается в том, что шаг адаптации нормализуется с учётем энергии входного сигнала. Это позволяет алгоритму автоматически подстраиваться под изменения амплитуды входа, делая обучение более стабильным.

\textbf{Обновление весов}
\begin{equation*}
  w\left(n+1\right)=w\left(n\right)+\frac{\mu}{\varepsilon+|x\left(n\right)|^2}\cdot e\left(n\right)\cdot x\left(n\right)
\end{equation*}
\noindent
Здесь шаг μ делится на норму входного вектора, чтобы избежать слишком больших изменений весов при сильном входном сигнале.

Это особенно важно при работе с сигналами, амплитуда которых может сильно меняться (ЭКГ с различными шумами как раз подходит под это определение). Также добавляется малый параметр $\varepsilon$ в знаменатель \textendash\ он нужен, чтобы избежать деления на ноль.
В результате NLMS сходится быстрее и устойчивее LMS, особенно в системах с переменными параметрами сигнала.

\subsubsection{1.2.3 RLS (Recursive Least Squares)}
Рекуррентный метод наименьших квадратов~\cite{dhas2022}, в отличие от LMS, минимизирует взвешенную сумму квадратов ошибок, что обеспечивает более быструю сходимость и устойчивость к коррелированным входным данным (ситуация, когда последовательные отсчёты входного сигнала статистически зависимы).

RLS рекуррентно обновляет коэффициенты фильтра, минимизируя целевую функцию:
\begin{equation*}
  J\left(n\right)=\sum_{i=1}^{n}{\lambda^{n-i}\left[e\left(i\right)\right]^2}
\end{equation*}
\noindent
Где 	$\lambda$ \textendash\ фактор забывания ($0<\lambda\le1$), который уменьшает влияние старых данных, 	$e\left(i\right)$  \textendash\ ошибка на шаге i.

Алгоритм использует матрицу ковариации для учета корреляции между входными данными, что позволяет быстрее адаптироваться к изменениям сигнала.

Формулы, описывающие его работу:

\textbf{Выход фильтра}
\begin{equation*}
  \hat{y}\left(n\right)=w^T\left(n-1\right)\thinspace x\left(n\right)
\end{equation*}

\textbf{Ошибка фильтрации}
\begin{equation*}
  e\left(n\right)=d\left(n\right)-\hat{y}\left(n\right)
\end{equation*}

\textbf{Вектор усиления}
\begin{equation*}
  k\left(n\right)=\frac{P\left(n-1\right)\thinspace x\left(n\right)}{\lambda+x^T\left(n\right)\thinspace P\left(n-1\right)\thinspace x\left(n\right)}
\end{equation*}
\noindent
Этот вектор определяет, насколько каждый компонент весов должен учесть новую информацию.

\textbf{Обновление весов}
\begin{equation*}
  w\left(n\right)=w\left(n-1\right)+k\left(n\right)\thinspace` e\left(n\right)
\end{equation*}

\textbf{Обновление обратной матрицы ковариации ошибок}
\begin{equation*}
  P\left(n\right)=\frac{1}{\lambda}\left[P\left(n-1\right)-k\left(n\right)\thinspace x^T\left(n\right)\thinspace P\left(n-1\right)\right]
\end{equation*}
\noindent
Корректировка $P\left(n\right)$ нужна, чтобы учесть новейшее наблюдение и забыть старые данные, в соответствии с заданным фактором забывания $\lambda$.

К преимуществам RLS можно отнести высокую скорость сходимости и минимизацию ошибки, с учётом всей истории данных. Также RLS более гибкий по сравнению с LMS, так как параметр $\lambda$ позволяет контролировать скорость забывания старых данных.

К его недостаткам относятся довольно высокая вычислительная сложность $(O\left(M^2\right)$ для фильтра порядка M) и чувствительность к выбору фактора забывания $\lambda$.

\subsubsection{1.2.4 Фильтр Калмана}
Фильтр Калмана~\cite{mneimneh2006} работает в два этапа.

1.	Предсказание: оценка текущего состояния системы на основе предыдущих данных.

2.	Коррекция: уточнение оценки с учетом новых измерений.

\noindent
В этом методе ЭКГ-сигнал описывается авторегрессионной моделью, а дрейф изолинии аппроксимируется полиномом низкого порядка, изменяющимся во времени. В такой постановке система представляется в виде модели состояния: коэффициенты авторегрессии отражают динамику ЭКГ, а коэффициенты полинома \textendash\ поведение дрейфа. На каждом шаге фильтр выполняет прогноз значений на основе предыдущего состояния и корректирует его с учётом текущего измерения, одновременно уточняя оценки параметров сигнала и дрейфа. В результате получается разделение исходного сигнала на полезную часть и медленно меняющуюся базовую линию.

Калмановский фильтр описывается следующими уравнениями:

\textbf{Модель перехода состояния}
\begin{equation*}
  x_k=Fx_{k-1}+w_k
\end{equation*}
\noindent
Где 	$x_k$ \textendash\ вектор состояния на шаге k (Истинный сигнал ЭКГ, дрейф изолинии), $F$ \textendash\ матрица перехода состояния, 	$w_k$ \textendash\ шум процесса (неучтённые компоненты).

\textbf{Модель наблюдения}
\begin{equation*}
  z_k=Hx_k+v_k
\end{equation*}
\noindent
Где 	$z_k$ \textendash\ вектор измерений (наблюдаемый зашумленный сигнал), $H$ \textendash\ матрица измерений (определяет какие компоненты состояния видны в измерениях), $v_k$ \textendash\ шум измерений.

\textbf{Этап предсказания}
\begin{equation*}
  {\hat{x}}_{k|k-1}\ =\ F{\hat{x}}_{k-1|k-1}
\end{equation*}
\begin{equation*}
P_{k|k-1}\ =\ FP_{k-1|k-1}F^T\ +\ Q
\end{equation*}
\noindent
Где 	${\hat{x}}_{k|k-1}$ \textendash\ предсказанное состояние, $P_{k|k-1}$ \textendash\ ковариация ошибки предсказания, $Q$ \textendash\ ковариационная матрица шума процесса.

\textbf{Этап коррекции}
\begin{equation*}
  K_k\ =\ P_{k|k-1}H^T{(HP_{k|k-1}H^T\ +\ R)}^{-1}
\end{equation*}
\begin{equation*}
  {\hat{x}}_{k|k}\ =\ {\hat{x}}_{k|k-1}+K_k\ (z_k\ -\ H{\hat{x}}_{k|k-1})
\end{equation*}
\begin{equation*}
  P_{k|k}\ =\ (I\ -\ K_kH)P_{k|k-1}
\end{equation*}
\noindent
Где 	$K_k$ \textendash\ коэффициент усиления Калмана, $R$ \textendash\ ковариационная матрица шума измерений, ${\hat{x}}_{k|k}$ \textendash\ обновлённое состояние, $P_{k|k}$ \textendash\ обновлённая ковариация, $I$ \textendash\ единичная матрица.

Для линейных систем с гауссовым шумом фильтр Калмана обеспечивает наилучшую оценку в смысле наименьших квадратов, минимизируя ошибку. Он хорошо справляется с шумами, если модель сигнала точна. Рекурсивная природа алгоритма позволяет обрабатывать данные в реальном времени. Он способен подстраиваться под изменяющиеся характеристики шума и сигнала, улучшая качество обработки.\\
К недостаткам можно отнести зависимость от модели. Если модель сигнала или шумов неточна, результат может быть плохим. Также стандартный фильтр предполагает линейность (модель перехода состояния и наблюдения представляют собой линейные преобразования), что может быть проблемой при обработке сигнала ЭКГ.

\subsubsection{1.2.5 Wavelet + LMS}
Метод~\cite{biswas2019} основан на применении дискретного вейвлет-преобразования (DWT) для локализации шумов в разных частотных диапазонах. На первом этапе сигнал ЭКГ разлагается на набор аппроксимирующих и детализирующих коэффициентов, что позволяет выделить как низкочастотные компоненты, соответствующие дрейфу изолинии, так и высокочастотные, связанные с помехами. На втором этапе к детализирующим коэффициентам применяется пороговая обработка: коэффициенты, амплитуда которых ниже выбранного порога, обнуляются или уменьшаются, поскольку они с высокой вероятностью соответствуют шуму. Далее адаптивная фильтрация корректирует оставшиеся коэффициенты, минимизируя влияние искажений при сохранении структуры исходного сигнала. Наконец, на третьем этапе выполняется обратное вейвлет-преобразование, в ходе которого из модифицированных коэффициентов формируется восстановленный ЭКГ-сигнал с подавленным дрейфом изолинии и другими артефактами.

\textbf{Дискретное вейвлет преобразование (DWT)}
\begin{equation*}
  x\left(t\right)=\sum_{k}{a_{j_0,k}\phi_{j_0,k}\left(t\right)}+\sum_{j=j_0}^{J}\sum_{k}{d_{j,k}\psi_{j,k}\left(t\right)}
\end{equation*}
\noindent
Где 	$a_{j_0,k}$ \textendash\ коэффициенты аппроксимации (низкие частоты), $d_{j,k}$ \textendash\ коэффициенты деталей (высокие частоты), $\phi$ и ψ \textendash\ масштабирующая и вейвлет-функции.

Вейвлеты выделяют локальные шумы (например, из-за движения пациента), а адаптивный фильтр подавляет их.\\
К достоинствам можно отнести следующие свойства этого метода.
Он хорошо сохраняет важные компоненты ЭКГ сигнала (QRS-комплексы).
Его можно комбинировать с любым адаптивным фильтром (LMS, RLS, NLMS).\\
К недостаткам можно отнести то, что результат зависит от типа вейвлет-функции; неправильный выбор может исказить сигнал. Метод требует настройки адаптивного фильтра. Требуется точная настройка шага обучения и порогов для вейвлет-коэффициентов, что может быть сложной задачей.

\subsubsection{1.2.6 EMD (Empirical Mode Decomposition) + LMS}
В этом методе~\cite{huang2015} мы сначала выполняем эмпирическую модальную декомпозицию (EMD) исходного сигнала $x(n)$ и получаем ряд из K IMF-компонент $c_k(n)$ и остаточный низкочастотный сигнал $r_k(n)$. Остаток $r_k(n)$ интерпретируется как приближение дрейфа базовой линии.

\textbf{Декомпозиция EMD}
\begin{equation*}
  x\left(n\right)=\sum_{k=1}^{K}{c_k\left(n\right)}+r_K\left(n\right)
\end{equation*}

\textbf{Оценка дрейфа изолинии}
\begin{equation*}
  b\left(n\right)=r_K\left(n\right)
\end{equation*}
Для выделения отдельных IMF-компонент выполняется следующее:
\begin{enumerate}
  \item Выделяются локальные минимумы и максимумы сигнала
  \item Через локальные максимумы и минимумы проводят две огибающие (т.е. сигнал ЭКГ находится между этими двумя кривыми)
  \item Берут среднее между верхней и нижней огибающими. Это создаёт приближение дрейфа изолинии, который отражает общий тренд сигнала.
  \item Вычитают это усреднение из исходного сигнала. Получается новый сигнал \textendash\ первая "внутренняя модовая функция" (IMF). Это быстрые колебания, вроде высокочастотного шума или мелких деталей ЭКГ.
  \item Повторяют шаги 1-4 до достижения заданного критерия остановки.
\end{enumerate}
Далее применяется LMS фильтр для уточнения дрейфа. И затем, используя оценку дрейфа в качестве опорного сигнала, сигнал реконструируется, но уже без дрейфа изолинии.\\
К достоинствам алгоритма можно отнести сравнительно легкую настройку (по сути нужно только выбрать количество компонент, на которые раскладывается сигнал) и хорошее сохранение важных QRS-комплексов, особенно высокочастотных). Также EMD можно комбинировать с любым адаптивным фильтром (не обязательно LMS).\\
Среди недостатков, высокая вычислительная сложность и чувствительность к сильным шумам.

\subsubsection{1.2.7 DFT + SSRLS (State Space Recursive Least Squares)}
Метод SSRLS~\cite{malik2004,sheikh2016} \textendash\ это адаптивный фильтр, основанный на методе рекурсивных наименьших квадратов (RLS), но реализованный в терминах состояний, как в фильтре Калмана.\\
Дрейф изолинии моделируется как низкочастотная синусоида (0.15–0.3 Гц). Сначала выполняется дискретное преобразование Фурье (ДПФ) для точной оценки частоты дрейфа. Эта частота используется для настройки матрицы перехода состояний $A[n]$, чтобы модель как можно более точно соответствовала характеристикам дрейфа.\\
Затем применяется SSRLS-фильтр, который итеративно оценивает вектор состояния $x[n]$, соответствующий текущей фазе и амплитуде дрейфа. Коэффициент усиления $K[n]$ обновляется аналогично фильтру Калмана, и вся система отслеживает изменения частоты в реальном времени.\\
После того как дрейф приближенно восстановлен, он вычитается из исходного сигнала, и получается очищенный сигнал ЭКГ.

К достоинствам этого метода относится то, что он не требует эталонного сигнала (в отличие от стандартных LMS, RLS фильтров), хорошо сохраняет морфологию сигнала.\\
К недостаткам относится высокая вычислительная сложность из-за итераций ДПФ и зависимость качества работы метода от формы шума (чем ближе она к синусоидальной, тем он лучше работает).

\newpage
\subsection{1.3 Сравнение методов и выводы}
В таблице 1 приведено сравнение методов по выделенным критериям на основе публикаций.

\noindent
\begin{tabularx}{\textwidth}{
	>{\raggedright\arraybackslash}p{1.5cm}
	>{\raggedright\arraybackslash}p{1.5cm}
	>{\raggedright\arraybackslash}p{1.5cm}
	>{\raggedright\arraybackslash}p{1.5cm}
	>{\raggedright\arraybackslash}p{1.5cm}
	>{\centering\arraybackslash}X
}
\toprule
\textbf{Метод} & \textbf{Улучшение SNR (дБ)} & \textbf{Искажения (\% RMSE)} & \textbf{Итерации до сходимости} & \textbf{Сложность на отсчёт} & \textbf{Наличие реализации}\\
\midrule
LMS & 3–6 & 5–10 & 5000–20000 & $O(m)$ & Есть (MATLAB, Python)\\
NLMS & 5–8 & 3–7 & 1000–5000 & $O(m)$ & Есть (MATLAB, Python)\\
RLS & 10–15 & 2–5 & 100–1000 & $O(m^2)$ & Есть (MATLAB, Python)\\
Фильтр Калмана & 8–12 & 3–6 & 200–2000 & $O(n^3)$ & Есть (MATLAB, Python, C++)\\
Wavelet + LMS & 15–20 & 1–3 & 5000–15000 & $O(J(L+M))$ & Есть (MATLAB, Python)\\
EMD + LMS & 10–18 & 2–4 & 2000–10000 & $O(K)$ & Частично (Python, MATLAB)\\
Intelligent DFT + SSRLS & 15–20 & 1–3 & 100–1000 & $O(n^3)$ & Частично (MATLAB)\\
\bottomrule
\end{tabularx}
{\centering Таблица 1: Сравнение методов удаления дрейфа изолинии}


В столбце «Сложность на отсчёт» $m$ \textendash\ число коэффициентов фильтра (размер окна), $n$ \textendash\ число компонент состояния (обычно мало: 2–4), $L$ \textendash\ длина вейвлет-фильтра (обычно 2–8), $J$ \textendash\ число уровней разложения (обычно 3–8), $K$ \textendash\ количество IMF, $E$ \textendash\ среднее число экстремумов ЭКГ-сигнала в окне.

Основываясь на обзоре, было решено принять к реализации метод из раздела 1.2.7 intelligent DFT + SSRLS.\\ 
Во-первых, SSRLS не требует наличия эталонного сигнала для удаления дрейфа базовой линии (Baseline Wander), что выгодно отличает его от классических методов адаптивной фильтрации, таких как LMS и RLS, которые нуждаются в референсном сигнале для коррекции. Во-вторых, для точной предварительной оценки частоты дрейфа (BLW) используется метод Intelligent DFT. Он работает в два этапа: сначала выполняется грубая оценка частоты с низким разрешением, затем \textendash\ уточнённая в узком диапазоне вокруг найденного пика. Полученное значение частоты используется для настройки матрицы перехода состояния SSRLS-фильтра, что обеспечивает его точную адаптацию к характеристикам дрейфа. Также этот метод обеспечивает высокоэффективное подавление низкочастотного дрейфа (большое улучшение SNR) при минимальном искажении полезного сигнала ЭКГ.\\
Кроме того, SSRLS демонстрирует высокую устойчивость и способность отслеживать изменения частоты BLW.

\section{2. Описание алгоритма}
В данной работе для реализации был выбран метод intelligent DFT + SSRLS~\cite{malik2004,sheikh2016}. В этом разделе он описан более детально.\\
Метод предназначен для удаления низкочастотного дрейфа изолинии из сигнала ЭКГ без использования эталонного сигнала. Он сочетает два ключевых компонента:

\begin{itemize}
  \item \textbf{Интеллектуальное дискретное преобразование Фурье (DFT)} \textendash\ для точной оценки частоты BLW в заданном диапазоне частот;
  \item \textbf{SSRLS} \textendash\ адаптивный фильтр в пространстве состояний, настраиваемый на оценённую частоту и удаляющий BLW.
\end{itemize}

На рисунке 1 изображена общая схема работы алгоритма.	
\medskip

\noindent
\begin{minipage}{\textwidth}
	\centering
    \includegraphics[width=\linewidth]{image2}
    \centering Рис. 1
    \label{fig:image2}
\end{minipage}
\medskip

\subsection*{2.1 Модель сигнала}
Зашумлённый сигнал ЭКГ описывается как $y[n] = s[n] + b[n] + w[n]$, где 	$y[n]$ \textendash\ зашумленный сигнал ЭКГ, $s[n]$ \textendash\ чистый сигнал ЭКГ, $b[n]$ \textendash\ BLW-шум, $w[n]$ \textendash\ белый шум. Дрейф $b[n]$ моделируется как $A \sin(2\pi f_0 n T_s + \varphi)$, где $A$ \textendash\ амплитуда, $f_0$ \textendash\ частота BLW (0.15–0.30~Гц), $T_s$ \textendash\ период дискретизации.

\subsection*{2.2 Интеллектуальное ДПФ}
Двухэтапная оценка частоты BLW: грубый поиск в диапазоне 0.1–0.3~Гц с шагом 0.02~Гц и точная дооценка в узком окне (например, 0.20–0.22~Гц) с шагом 0.002~Гц. Для подавления утечки спектра у краёв выбранного сегмента сигнала применяется окно Хэннинга.

\subsection*{2.3 SSRLS-фильтр}
Модель в пространстве состояний:
\begin{gather*}
    x\left[n+1\right]=Ax\left[n\right] \hspace{3cm} y\left[n\right]=Cx\left[n\right]+v\left[n\right]
\end{gather*}
\noindent
Где 	$x\left[n\right]$ \textendash\ вектор состояния, $A$ \textendash\ матрица перехода состояния (синусоидальная модель), 	 $C\ =\ [1\ 0]$ \textendash\ матрица измерений.

Матрица перехода:
\begin{equation*}
  A\ =\ \left[\begin{matrix}cos\left(wT\right)&sin\left(wT\right)\\-sin\left(wT\right)&cos\left(wT\right)\\\end{matrix}\right]
\end{equation*}

Где $w=2\pi f$ \textendash\ угловая частота BLW, оцененная с помощью ДПФ. Матрица $A$ обеспечивает вращение вектора состояния с частотой $f$.\\
\textbf{Алгоритм SSRLS}

1. Предсказание состояния:  
   \[
   \hat{x}\left[n\right]=A\hat{x}\left[n-1\right]
   \]

2. Коррекция:  
   \[
   K\left[n\right]=\Phi^{-1}\left[n\right]{\ C}^T - \text{обновление коэффициента Калмана}
   \]
   \[
   \Phi\left[n\right]=\lambda A^{-T}\Phi\left[n-1\right]A^{-1}+C^TC - \text{обновление матрицы ковариации ошибок}
   \]  
   Где $\lambda$ \textendash\ коэффициент забывания.

3. Обновление оценки:  
   \[
   \hat{x}\left[n\right]=\hat{x}\left[n\right]+K\left[n\right]\left(y\left[n\right]-C\hat{x}\left[n\right]\right)
   \]

\textbf{Очищенный сигнал}
   \[
   \hat{s}\left[n\right]=y\left[n\right]-C\hat{x}\left[n\right]
   \]

\section{3. Экспериментальное исследование}
\subsection{3.1 Цели исследования}
\begin{enumerate}
  \item Выявление зависимости между отношением сигнал/BLW и качеством очистки сигнала ЭКГ. Оценка качества определения частоты дрейфа BLW с помощью ДПФ. Метрики качества: MSE \textendash\ среднеквадратическая ошибка (отклонение отфильтрованного сигнала ЭКГ от исходного сигнала без шума BLW.
  \item Визуальная оценка сохраняемости элементов кардиокомплекса после фильтрации.
  \item Определение скорости сходимости метода.
  \item Измерение процессорного времени работы метода удаления шума BLW.
\end{enumerate}

\subsection{3.2 Объект исследования}
Объектом исследования служит программа на языке Python \textendash\ реализация метода
фильтрации BLW на основе адаптивного метода наименьших квадратов в пространстве состояний динамической системы.

\subsection{3.3 Методика исследования}
Эксперименты делятся на два типа: со сгенерированным шумом (постоянные характеристики) и с реальным шумом.
\paragraph{Синтетический BLW.} Генерируется синусоидальный шум $b[n] = A \sin(2\pi f T)$, накладывается на чистый сигнал, выполняется фильтрация. Оцениваются MSE, точность оценки частоты, число отсчётов до сходимости, влияние $A$ и $f$ на качество фильтрации.
\paragraph{Реальный BLW.} Используются реальные записи и шумовые компоненты; оцениваются MSE и визуальная сохранность морфологии ЭКГ.

\subsection{3.4 Результаты экспериментов}
Будем использовать обозначения $A$ \textendash\ амплитуда BLW, $f$ \textendash\ частота BLW. На один и тот же чистый сигнал ЭКГ был наложен один и тот же шум ($f=0{,}16$~Гц, $A=0{,}15$~мВ), но в первом случае (Рис. 2) использовано 10 секунд сигнала, а во втором (Рис. 3) 60 секунд.

\noindent
\begin{minipage}[t]{0.48\textwidth}
    \centering
    \includegraphics[width=\linewidth]{image3}
    \centering Рис. 2\\MSE 0.00064, оценённая частота BLW 0.156 Гц
    \label{fig:image3}
\end{minipage}
\hfill
\begin{minipage}[t]{0.48\textwidth}
    \centering
    \includegraphics[width=\linewidth]{image4}
    \centering Рис. 3\\MSE 0.00022, оценённая частота BLW 0.160 Гц
    \label{fig:image4}
\end{minipage}


\medskip
Как видно из графиков, в обоих случаях наш метод довольно хорошо оценил частоту шума и убрал синтетический шум (MSE<0.001). Для сигнала длиной 60 секунд MSE в 3 раза меньше. Это подтверждает сходимость метода. Это хорошо видно на следующем графике (Рис. 4), где изображена разница между реальным значением шума и предсказанным.

\noindent
\begin{minipage}{0.60\textwidth}
	\centering
    \includegraphics[width=\linewidth]{image5.png}
    \centering Рис. 4
    \label{fig:image5}
\end{minipage}
\hfill
\begin{minipage}{0.36\textwidth}
        Видно, что алгоритм сходится примерно через 5 секунд. В нашем случае частота дискретизации 250 Гц, а значит алгоритму нужно порядка 1250 отсчётов, чтобы сойтись.
\end{minipage}
\medskip

Исследуем влияние частоты и амплитуды BLW на качество фильтрации.
Для одного и того же чистого сигнала ЭКГ были взяты четыре варианта шума:
\begin{enumerate}
  \item $f=0{,}15$~Гц, $A=0{,}2$~мВ (Рис. 5);
  \item $f=0{,}15$~Гц, $A=0{,}6$~мВ (Рис. 6);
  \item $f=0{,}3$~Гц, $A=0{,}2$~мВ (Рис. 7);
  \item $f=0{,}3$~Гц, $A=0{,}6$~мВ (Рис. 8);
\end{enumerate}

\noindent
\begin{minipage}[t]{0.48\textwidth}
    \centering
    \includegraphics[width=\linewidth]{image6}
    \centering Рис. 5\\MSE 0.00043
    \label{fig:image6}
\end{minipage}
\hfill
\begin{minipage}[t]{0.48\textwidth}
    \centering
    \includegraphics[width=\linewidth]{image7}
    \centering Рис. 6\\MSE 0.0010
    \label{fig:image7}
\end{minipage}

\noindent
\begin{minipage}[t]{0.48\textwidth}
    \centering
    \includegraphics[width=\linewidth]{image8}
    \centering Рис. 7\\MSE 0.00057
    \label{fig:image8}
\end{minipage}
\hfill
\begin{minipage}[t]{0.48\textwidth}
    \centering
    \includegraphics[width=\linewidth]{image9}
    \centering Рис. 8\\MSE 0.0023
    \label{fig:image9}
\end{minipage}
\medskip

Как видно из графиков выше увеличение частоты и особенно амплитуды BLW ухудшают качество работы нашего метода.
 
Проведем следующий эксперимент. Выделим набор файлов ЭКГ (с чистым ЭКГ сигналом), предоставленными ООО «Нейрософт» (15 произвольно выбранных файлов). Построим набор шумовых сигналов BLW. Частота шума принимается равной 0.15; 0.2; 0.25; 0.3 Гц. Амплитуда шума устанавливается 0.5; 1; 2 мВ. Т.е. на каждый чистый сигнал ЭКГ будет накладываться 12 вариантов шума.

Проведем серию замеров, накладывая шум на сигнал перед фильтрацией и вычисляя MSE. Также вычислим отношение сигнал-шум для сигнала с наложенным шумом. Построим график зависимости MSE от SNR

Результат виден на следующем графике (Рис. 9) зависимости MSE после фильтрации от SNR до удаления шума.

\noindent
\begin{minipage}{\textwidth}
	\centering
    \includegraphics[width=\linewidth]{image10.png}
    \centering Рис. 9
    \label{fig:image10}
\end{minipage}
\medskip

Явно видно, что при увеличении SNR (ослабление шума BLW) алгоритм работает лучше. Но даже при SNR, близком к нулю (мощности шума и чистого сигнала сопоставимы), синтетический шум удаляется хорошо.

Также было замерено процессорное время работы алгоритма для каждого запуска. Среднее время $0{,}22$~с на 60~с сигнала при частоте дискретизации 250~Гц, то есть примерно $0{,}15$~с на $10^4$ отсчётов. Замер производился на устройстве с 16-ядерным процессором AMD Ryzen 7 7840HS (3{,}80~ГГц). Т.е. алгоритм работает достаточно быстро, чтобы обрабатывать ЭКГ сигнал в реальном времени.

Теперь рассмотрим несколько экспериментов с реальными шумами BLW. Для примера выбраны три варианта реального шума: два из них хорошо аппроксимируются синусоидой, один \textendash\ плохо.

\noindent
\begin{minipage}[t]{0.48\textwidth}
    \centering
    \includegraphics[width=\linewidth]{image11}
    \centering Рис. 10\\MSE 0.0028
    \label{fig:image11}
\end{minipage}
\hfill
\begin{minipage}[t]{0.48\textwidth}
    \centering
    \includegraphics[width=\linewidth]{image12}
    \centering Рис. 11\\MSE 0.0041
    \label{fig:image12}
\end{minipage}

\noindent
\begin{minipage}{\textwidth}
	\centering
    \includegraphics[width=0.80\linewidth]{image13}
    \par
    \centering Рис. 12\\MSE 0.037
    \label{fig:image13}
\end{minipage}
\medskip

\noindent
Как видно из приведенных выше графиков, шум, который плохо аппроксимируется синусоидой, плохо фильтруется и показал MSE 0.037 для 30 секунд сигнала.

\section{4. Заключение}
Экспериментальное исследование показало, что на качество удаления шума изолинии с помощью алгоритма intelligent DFT + SSRLS прежде всего влияет форма этого шума. Чем лучше он аппроксимируется синусоидой, тем чище будет отфильтрованный сигнал. Также на качество работы реализованного алгоритма влияют амплитуда и частота шума. Чем они больше, тем больше ошибка фильтрации. Также было замечено, что для сходимости алгоритма (стабилизации отклонения от истинного сигнала) нужно около 1250–1500 отсчётов.\\
Также очень важным аспектом является реальная скорость работы алгоритма. Эксперименты показали, что он работает достаточно быстро, чтобы обрабатывать ЭКГ сигнал в реальном времени.

\begin{thebibliography}{10}
\bibitem{worldhealth} World Health Organization. (2025, July 31). Cardiovascular diseases (CVDs). Retrieved (2025, September 07), from \url{https://www.who.int/news-room/fact-sheets/detail/cardiovascular-diseases-(cvds)}
\bibitem{pandey2010} V.~K.~Pandey. Adaptive filtering for baseline wander removal in ECG. In: \textit{10th IEEE International Conference on Information Technology and Applications in Biomedicine}, Corfu, 2010, pp.~1--4.
\bibitem{rehman2012} S.~A.~Rehman, R.~Kumar, M.~Rao. Performance comparison of various adaptive filter algorithms for ECG signal enhancement and baseline wander removal. In: \textit{Third International Conference on Computing, Communication and Networking Technologies}, 2012.
\bibitem{dhas2022} D.~Edwin~Dhas, M.~Suchetha. Dual phase dependent RLS filtering approach for baseline wander removal in ECG signal acquisition. \textit{Biomedical Signal Processing and Control}, 77, 2022.
\bibitem{mneimneh2006} M.~A.~Mneimneh. An adaptive Kalman filter for removing baseline wandering in ECG signals. \textit{Computers in Cardiology}, 2006, pp.~253--256.
\bibitem{biswas2019} U.~Biswas, M.~Maniruzzaman, B.~Sana, K.~R.~Hasan. Removing Baseline Wander from ECG Signal Using Wavelet Transform. \textit{Khulna University Studies}, 16(1\&2), 2019, pp.~61--73.
\bibitem{huang2015} W.~Huang, N.~Cai, W.~Xie, Q.~Ye, Z.~Yang. ECG Baseline Wander Correction Based on Ensemble Empirical Mode Decomposition with Complementary Adaptive Noise. \textit{Journal of Medical Imaging and Health Informatics}, 5(7), 2015, pp.~1547--1554.
\bibitem{malik2004} M.~B.~Malik. State-space recursive least-squares: Part I. \textit{Signal Processing}, 84, 2004, pp.~1709--1718.
\bibitem{sheikh2016} S.~A.~Sheikh et al. Baseline wander removal from ECG signal using State Space Recursive Least Squares (SSRLS) adaptive filter. In: \textit{International Conference on Robotics and Artificial Intelligence (ICRAI)}, 2016.
\end{thebibliography}

\end{document}
